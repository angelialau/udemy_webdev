\documentclass[11pt, a4paper]{article}   	% use "amsart" instead of "article" for AMSLaTeX format
\usepackage{geometry}                		% See geometry.pdf to learn the layout options. There are lots.
\geometry{a4paper}                   		% ... or a4paper or a5paper or ... 
%\geometry{landscape}                		% Activate for rotated page geometry
\usepackage[parfill]{parskip}    		% Activate to begin paragraphs with an empty line rather than an indent
\usepackage{graphicx}				% Use pdf, png, jpg, or eps§ with pdflatex; use eps in DVI mode
								% TeX will automatically convert eps --> pdf in pdflatex		
\usepackage{amssymb, amsmath}

\title{Learning about JavaScript}
\author{Angelia Lau}
\date{\today}	
\newcommand\tab[1][1cm]{\hspace*{#1}} %declaraction for \tab command

\begin{document}
\maketitle

\section{Primitive Datatypes}
\begin{enumerate}
	\item Numbers
	\item Strings
	\item Boolean
	\item Null (explicitly defined as nothing)
	\item Undefined	(when a variable is declared but not initialised to a value yet)
\end{enumerate}

\subsection{Numbers}
You can do math in the console! 

\subsection{Strings }
\begin{itemize}
	\item Single quotes and double quotes are treated equally, as long as they are matched! 
	\item Concatenation works 
	\item Escape characters: \textbackslash " or \textbackslash '
	\item ``A string''.length returns len %use '`' to write left quotation marks
	\item ``The Beatles"[5] gives ``e"
\end{itemize}

\subsection{Quick Exercise}
\begin{enumerate}
	\item $100 \mathbin{\%} 3 = 1$
	\item (``blah" + ``blah")[6] = ``a"
	\item $``hello".length \mathbin{\%} ``hi\backslash".length = 5 \% 3 = 2$
\end{enumerate}

\section{Variables in JS}
\begin{enumerate}
	\item var name = whatever 
	\item var names have to be camelCase
\end{enumerate}


\section{Methods in JS}
\begin{enumerate}
	\item clear() is an example method, it clears the console
	\item Other built in functions: 
	\begin{itemize}
		\item \texttt{alert(``Hello there!")} shows a pop up 
		\item \texttt{console.log(``hello from the console")} is like a LogCat message in AS
		\item \texttt{var userName = prompt(``What is your name?");} \\ stores user input to the variable userName
	\end{itemize}
\end{enumerate}

\section{Boolean logic in JS}

Apart from the usual $>$, $>=$, $<$, $<=$, $\&\&$, $\mid\mid$, $!$ there are four more:

\begin{center}
\texttt{Assuming $x=5$,}

\begin{tabular}{|c|c|c|c|}
	\hline
	Operator & Name & Example & Result\\ \hline
	== & Equal to & x==``5" & true\\ \hline
	$\!$= & Not equal to & x$\!$=9 & true\\ \hline
	=== & Equal value and type & x===``5" & false\\ \hline
	$\!==$ & Not equal value and type & x$\!$==``5" & true\\ \hline
\end{tabular}

\emph{note that ``==" does type coercion, and converts different data types to the same one }

Here are some interesting cases:

\begin{tabular}{|c|c|}
	\hline
	Example & Result\\ \hline
	true == ``1" & true\\ \hline
	false == ``0" & true\\ \hline
	null == undefined & true\\ \hline
	NaN == NaN & false\\ \hline
\end{tabular}

\emph{note that NaN = Not A Number}
\end{center}

\subsection{Truthy and Falsey}
Inherently \emph{falsey} values; false, 0, "", null, undefined, NaN. \\Everything else is \emph{truthy}.

\subsection{Quick Exercise}
\subsubsection{Question 1}
	\begin{align*}
		\text{var x} &= 10;\\
		\text{var y} &= ``a";\\
		(y === ``b") \mid\mid (x>=10) &= true
	\end{align*}
	
\subsubsection{Question 2}
	\begin{align*}
		\text{var x} &= 3;\\
		\text{var y} &= 8;\\
		![(x==``3")\mid\mid(x===y)] \hspace{4pt}\&\& \hspace{4pt} ![(y!=8) \&\& (x<=y)]  &= false
	\end{align*}
	
\subsubsection{Question 3}
	\begin{align*}
		\text{var str} &= ``";\\
		\text{var msg} &= ``haha!";\\
		\text{var isFunny} &= ``false";\\
		!((str \mid\mid msg) \hspace{4pt}\&\&\hspace{4pt} isFunny) &= false
	\end{align*}
	
	Both parts are true since they have values, invert with $!$ and so it becomes false.
	
\section{Loops}
while and for loops are just like in java. Go rock em!

\section{Functions}
\subsection{Function Declaration}
\subsubsection{Example syntax}
\texttt{function capitalize(str)\{ \\
\tab return str.charAt(0).toUpperCase() + str.slice(1); \\
\}
}
\subsection{Function Expression}
\subsubsection{Example syntax}
\texttt{var capitalize=function(str)\{ \\
\tab return str.charAt(0).toUpperCase() + str.slice(1); \\
\}
}

However, like all variables, when the var value is changed to a normal value (ie a string or integer for example), the function expression will be lost. 


\section{Scope}
Every function has its own scope and the contents within the scope are not shared between functions. If variables within a function are not initialised with \texttt{var}, they will access global vars. Else, they will take on the initialised value. 
\subsection{Example:}
\texttt{
  var phrase = "hi there$!$" \\
  function doSomething()\{ \\
  $\-$ var phrase = ``Goodbye$!$''; \\
  $\-$ console.log(phrase); \\
  \}
}

When you enter the function \texttt{doSomething()}, it returns \texttt{Goodbye!}. But globally, when you search for \texttt{phrase}, you will get \texttt{"hi there!"}.


\section{Higher order functions!}
Passing functions to other functions. 
\subsection{Example}
\begin{center}
  \texttt{setInterval(<\emph{some function}>, <\emph{time interval in ms}>)}
\end{center}
So, an example would be \text{setInterval(singTwinkle, 1000)} which calls the \texttt{singTwinkle} method for 1000ms. Note that the function called doesn't have parenthesis. This is because the method isn't called by us, but called by the higher order method.

\subsection{Anonymous functions}
All of this can also be written as \texttt{setInterval(function()\{\emph{code here}$\dots$;\})}


\end{document}  